%Question 2


\begin{enumerate}
	
	%Part a
	\item{
	\begin{enumerate}
		%Part i
		\item{
			We know that in a Thevenin equivalent circuit, maximum power transfer to the load occurs when $Z_L = Z_{Th}^*$, or for entirely resistive circuits, $R_L = R_{Th}$.
			
			\begin{figure}[H]
				\centering
				\begin{circuitikz}
					\draw (0,3) to[vsource, l_=$v_{Th}$] (0,0);
					\draw (0,3) to[R=$R_{Th}$, -o] (3,3) -- 
						(4,3) to[R = $R_L$] (4,0)
							to[short, -o] (3,0) -- (0,0);
				\end{circuitikz}
				\caption{Thevenin equivalent circuit}
			\end{figure}

			Now by the formula sheet:
			\begin{align*}
				R_{Th} &= \frac{a_{12} + a_{22} \cdot R_g}{a_{11} + a_{21} \cdot R_g} \\
				&= \frac{10 + 1.5 \cdot 2}{4 + 0.5 \cdot 2} \\
				&= 2.6 \ \Omega
			\end{align*}
			Therefore, when $R_L = 2.6 \ \Omega$, maximum power is transferred to the load resistor.
			\\
		}
		
		%Part ii
		\item{
			Maximum power transferred to load can be found from the formula:
			\begin{equation*}
				P_{L \ max} = \frac{v_L^2}{R_L}
			\end{equation*}

			For the Thevenin equivalent circuit, $v_L$, the voltage drop across the load, can be found by voltage division. However, since $R_L = R_{Th}$, we know half of $v_{Th}$ will drop across $R_L$, the other half dropped across $R_{Th}$.
			\\ \\
			Now by the formula sheet:
			\begin{align*}
				v_{Th} &= \frac{v_g}{a_{11} + a_{21} \cdot R_g} \\
				&= \frac{10}{4 + 0.5 \cdot 2} \\
				&= 2 \ \text{V}
			\end{align*}
			\begin{align*}
				\therefore v_L &= \frac{v_{Th}}{2} \\
				&= 1 \ \text{V}
			\end{align*}
			\begin{align*}
				\therefore P_{L \ max} &= \frac{1^2}{2.6} \\
				&= 384.62 \ \text{mW}
			\end{align*}
			The maximum power delivered to the load is $P_{L \ max} = 384.62$ mW.
			\\
		}
		
		%Part iii
		\item{
			From (2a ii), $R_L = 2.6 \ \Omega \quad \therefore v_L = 1 \ \text{V}$. \\
			From this, $i_L$ can be found by Ohm's law:
			\begin{align*}
				i_L &= \frac{v_L}{R_L} \\
				&= \frac{1}{2.6} \\
				&= 384.62 \ \text{mA}
			\end{align*}
			Now by the formula sheet:
			\begin{equation*}
				\frac{i_2}{i_1} = \frac{-1}{a_{21} \cdot R_L + a_{22}}
			\end{equation*}
			And from this, $i_1$ can be found from $i_L$, where we note $i_2 = -i_L$:
			\begin{align*}
				i_1 &= -i_2 \left(a_{21} \cdot R_L + a_{22} \right) \\
				&= i_L \left(a_{21} \cdot R_L + a_{22} \right) \\
				&= 384.62 * 10^{-3} \left(0.5 \cdot 2.6 + 1.5 \right) \\
				&= 1.077 \ \text{A}
			\end{align*}
			And so, the current flowing into port 1 is $i_1 = 1.077$ A.
			\\
		}
	\end{enumerate}

	}
	
	%Part b
	\item{
		For measurement 1, we have the constraint that $V_2 = 0$ V. \\
		For measurement 2, we have the constraing that $V_1 = 0$ V. \\
		\\
		Therefore, we want to find a set of two-port network parameters that have $V_1$ and $V_2$ as the independent variables. \\
		\\
		We find the y parameters meet this condition:
		\begin{alignat*}{2}
			i_1 = y_{11} \cdot v_1 &+ y_{12} \cdot v_2 &\qquad \qquad \qquad \qquad \qquad
			i_2 = y_{21} \cdot v_1 &+ y_{22} \cdot v_2 \\ \\
			y_{11} = \frac{i_1}{v_1} &\Bigg|_{v_2 = 0} \ \text{S} &\qquad \qquad \qquad \qquad \qquad
			y_{21} = \frac{i_2}{v_1} &\Bigg|_{v_2 = 0} \ \text{S} \\
			y_{12} = \frac{i_1}{v_2} &\Bigg|_{v_1 = 0} \ \text{S} &\qquad \qquad \qquad \qquad \qquad
			y_{22} = \frac{i_2}{v_2} &\Bigg|_{v_1 = 0} \ \text{S} \\
		\end{alignat*}
		By measurement 1:
		\begin{equation*}
			y_{11} = \frac{1}{10} = 100 \ \text{mS} \qquad \qquad \qquad \qquad \qquad
			y_{21} = \frac{-0.5}{10} = -50 \ \text{mS}
		\end{equation*}
		\\ By measurement 2:
		\begin{equation*}
			y_{12} = \frac{-1}{20} = -50 \ \text{mS} \qquad \qquad \qquad \qquad \qquad
			y_{22} = \frac{3}{20} = 150 \ \text{mS} \\
		\end{equation*}
		Now for cascade of two-port networks, $[A_T] = [A_1] \times [A_2]$, where these are the a parameter matrices for the overall network, network 1 and network 2 respectively. \\
		\\
		Therefore convert above y parameters to a parameters to make finding overall network parameters easier (note $\Delta [Y]$ is the discriminant of the y parameter matrix):
		\begin{alignat*}{2}
			&a_{11} = -\frac{y_{22}}{y_{21}} = 3 &\qquad \qquad \qquad \qquad \qquad
			&a_{12} = -\frac{1}{y_{21}} = 20 \ \Omega \\
			&a_{21} = -\frac{\Delta [Y]}{y_{21}} = 0.25 \ \text{S} &\qquad \qquad \qquad \qquad \qquad
			&a_{22} = -\frac{y_{11}}{y_{21}} = 2
		\end{alignat*}
		As stated above, we can find the a parameters of the overall two-port network by matrix multiplying the a parameters of the constituent cascaded two-port networks:
		\begin{align*}
			[A_T] &= [A_1] \times [A_2] \\
			&= 
				\begin{bmatrix}
					4 & 10 \ \Omega \\
					0.5 \ \text{S} & 1.5
				\end{bmatrix}
				\begin{bmatrix}
					3 & 20 \ \Omega \\
					0.25 \ \text{S} 2
				\end{bmatrix}
			\\
			&=
				\begin{bmatrix}
					14.5 & 100 \ \Omega \\
					1.875 \ \text{S} & 13
				\end{bmatrix}
		\end{align*}
		And so we find that the a parameters of the cascaded network are $a_{11} = 14.5$, $a_{12} = 100 \ \Omega$, $a_{21} = 1.875 \ \text{S}$ and $a_{22} = 13$.
		\\
	}
\end{enumerate}