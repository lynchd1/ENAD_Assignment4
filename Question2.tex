%Question 2


\begin{enumerate}
	
	%Part a
	\item{
	\begin{enumerate}
		%Part i
		\item{
			We know that in a Thevenin equivalent circuit, maximum power transfer to the load occurs when $Z_L = Z_{Th}^*$, or for entirely resistive circuits, $R_L = R_{Th}$.
			
			\begin{figure}[H]
				\centering
				\begin{circuitikz}
					\draw (0,3) to[american voltage source, l_=$V_{Th}$] (0,0);
					\draw (0,3) to[R=$R_{Th}$, -o] (3,3) -- 
						(4,3) to[R = $R_L$] (4,0)
							to[short, -o] (3,0) -- (0,0);
				\end{circuitikz}
				\caption{Thevenin equivalent circuit}
			\end{figure}

			Now by the formula sheet:
			\begin{align*}
				R_{Th} &= \frac{a_{12} + a_{22} \cdot R_g}{a_{11} + a_{21} \cdot R_g} \\
				&= \frac{10 + 1.5 \cdot 2}{4 + 0.5 \cdot 2} \\
				&= 2.6 \ \Omega
			\end{align*}
			Therefore, when $R_L = 2.6 \ \Omega$, maximum power is transferred to the load resistor.
			\\
		}
		
		%Part ii
		\item{
			Maximum power transferred to load can be found from the formula:
			\begin{equation*}
				P_{L \ max} = \frac{V_L^2}{R_L}
			\end{equation*}

			For the Thevenin equivalent circuit, $V_L$, the voltage drop across the load, can be found by voltage division. However, since $R_L = R_{Th}$, we know half of $V_{Th}$ will drop across $R_L$, the other half dropped across $R_{Th}$.
			\\ \\
			Now by the formula sheet:
			\begin{align*}
				V_{Th} &= \frac{V_g}{a_{11} + a_{21} \cdot R_g} \\
				&= \frac{10}{4 + 0.5 \cdot 2} \\
				&= 2 \ \text{V}
			\end{align*}
			\begin{align*}
				\therefore V_L &= \frac{V_{Th}}{2} \\
				&= 1 \ \text{V}
			\end{align*}
			\begin{align*}
				\therefore P_{L \ max} &= \frac{1^2}{2.6} \\
				&= 384.62 \ \text{mW}
			\end{align*}
			The maximum power delivered to the load is 384.62 mW.
			\\
		}
		
		%Part iii
		\item{
		
		}
	\end{enumerate}
	}
	
	%Part b
	\item{
	
	}
\end{enumerate}