%Question 4

Design this bandreject filter by connecting a low pass Butterworth filter with a cutoff frequency of 500 Hz and gain of 0 dB in the passband in parallel with a high pass Butterworth filter with a cutoff frequency of 5 kHz and gain of 0 dB in the passband (thereby rejecting frequencies between 500 Hz and 5 kHz), then connecting both to separate inputs of a summing amplifier with gain of 20 dB in the passband.\\
\\
Start with designing the low pass stage.\\
The 3$^\text{rd}$ order normalised Butterworth polynomial is:
\begin{equation*}
	\big(s+1 \big) \big(s^2+s+1 \big)
\end{equation*}
Therefore the 3$^\text{rd}$ order normalised low pass Butterworth filter has the following transfer function:
\begin{equation*}
	\frac{1}{\big(s+1 \big) \big(s^2+s+1 \big)} = \frac{1}{\, s+1 \,} \cdot \frac{1}{\, s^2+s+1 \,}
\end{equation*}
This results in the cascade of a 1$^\text{st}$ order low pass Butterworth filter with a 2$^\text{nd}$ order low pass Butterworth filter, both with passband gains of 0 dB and cutoff frequencies of 1 rad/s. This can be achieved with the following circuit:
\begin{figure}[H]
	\centering
	\begin{circuitikz}
		\draw (0,2.5) to [R=$R_1$] (2,2.5)
			to [short] (2,4)
			to [R=$R_2$] (5,4)
			to [short] (5,2);
		\draw (2,4) to [short, *-] (2,5.5)
			to [C=$C_1$] (5,5.5)
			to [short, -*] (5,4);
		\draw (0,0) to [short] (13.5,0);
		\draw (0,2.5) to [open, v=$v_i$, o-o] (0,0);
		\draw (3.5,2) node[op amp] (opamp) {}
			(opamp.-) to [short, -*] (2,2.5)
			(opamp.+) to [short] (2,1.5)
				to [short, -*] (2,0)
			(opamp.out) to [short, -*] (5,2);
		\draw (5,2) to [short] (5.5,2)
			to [R = $R_3$] (7.5,2)
			to [R=$R_4$] (9.5,2)
			to [C=$C_3$, -*] (9.5,0);
		\draw (7.5,2) to [C=$C_2$, *-] (7.5,4.5)
			to [short] (9.5,4.5)
			to [short] (9.5,3);
		\draw (9.5,4.5) to [short, *-] (12.5,4.5)
			to [short] (12.5,2.5);
		\draw (11,2.5) node[op amp] (opamp) {}
			(opamp.-) to [short] (9.5,3)
			(opamp.+) to [short, -*] (9.5,2)
			(opamp.out) to [short, -*] (12.5,2.5);
		\draw (12.5,2.5) to [short] (13.5,2.5);
		\draw (13.5,0) to [open, v<=$v_o$, o-o] (13.5,2.5);
	\end{circuitikz}
\end{figure}
We need to find resistor and capacitor values for the normalised filters.\\
\\
Let $R_1 = R_2 = R_3 = R_4 = 1 \Omega$.\\
For the normalised 1$^\text{st}$ order low pass Butterworth filter, we note that the transfer function is the same as the transfer function of the prototypical 1$^\text{st}$ order active low pass filter. Therefore, the component values for this normalised Butterworth filter should match those of the 1$^\text{st}$ order prototype filter, and so $C_1 = 1$ F.\\
\\
For the normalised 2$^\text{nd}$ order low pass Butterworth filter, we note that the transfer function is:
\begin{align*}
	H_n(s) &= \frac{1/(C_2 \, C_3)}{s^2 + \frac{2}{C_2} \cdot s + \frac{1}{\, C_2 \, C_3 \,}} \\
	&= \frac{1}{\, s^2 + s + 1 \,} \\
	\\
	\therefore \frac{2}{C_2} &= 1 \\
	\therefore C_2 &= 2 \, \text{F} \\
	\\
	\therefore \frac{1}{C_2 \, C_3} &= 1 \\
	\therefore C_3 &= \frac{1}{C_2} \\
	&= 0.5 \, \text{F}
\end{align*}
With these above values, we have a 3$^\text{rd}$ order low pass Butterworth filter with a gain of 0 dB in the passband and a cutoff frequency of 1 rad/s.\\
\\
We now need to adjust the filter to get the desired cutoff frequency of 500 Hz (desired gain will be achieved by the gain stage).\\
First off we perform frequency scaling by finding the frequency scaling factor $k_f$:
\begin{align*}
	k_f &\triangleq \frac{\omega_c^{'}}{\omega_c} \\
	&= \frac{2 \cdot \pi \cdot 500}{1} \\
	&= 3141.6
\end{align*}
Now we must use 1 k$\Omega$ resistors in the final design of the low pass filter section, so we perform magnitude scaling by finding the magnitude scaling factor $k_m$:
\begin{align*}
	k_m &\triangleq \frac{R^{'}}{R} \\
	&= \frac{1000}{1} \\
	&= 1000
\end{align*}
We can now scale the capacitors to new values to achieve the desired cutoff frequency:
\begin{align*}
	C_n^{'} &= \frac{C_n}{\, k_m \, k_f \,} \\
	\\
	\therefore C_1^{'} &= \frac{1}{1000 \cdot 3141.6} = 318 \, \text{nF} \\
	\therefore C_2^{'} &= \frac{2}{1000 \cdot 3141.6} = 637 \, \text{nF} \\
	\therefore C_3^{'} &= \frac{0.5}{1000 \cdot 3141.6} = 159 \, \text{nF}
\end{align*}
And so we have finished the design of the low pass section, whose circuit is shown below:
\begin{figure}[H]
	\centering
	\begin{circuitikz}
		\draw (0,2.5) to [R=$1 \, \text{k}\Omega$] (2,2.5)
			to [short] (2,4)
			to [R=$1 \, \text{k}\Omega$] (5,4)
			to [short] (5,2);
		\draw (2,4) to [short, *-] (2,5.5)
			to [C=$318 \, \text{nF}$] (5,5.5)
			to [short, -*] (5,4);
		\draw (0,0) to [short] (13.5,0);
		\draw (0,2.5) to [open, v=$v_i$, o-o] (0,0);
		\draw (3.5,2) node[op amp] (opamp) {}
			(opamp.-) to [short, -*] (2,2.5)
			(opamp.+) to [short] (2,1.5)
				to [short, -*] (2,0)
			(opamp.out) to [short, -*] (5,2);
		\draw (5,2) to [short] (5.5,2)
			to [R = $1 \, \text{k}\Omega$] (7.5,2)
			to [R=$1 \, \text{k}\Omega$] (9.5,2)
			to [C=$159 \, \text{nF}$, -*] (9.5,0);
		\draw (7.5,2) to [C=$637 \, \text{nF}$, *-] (7.5,4.5)
			to [short] (9.5,4.5)
			to [short] (9.5,3);
		\draw (9.5,4.5) to [short, *-] (12.5,4.5)
			to [short] (12.5,2.5);
		\draw (11,2.5) node[op amp] (opamp) {}
			(opamp.-) to [short] (9.5,3)
			(opamp.+) to [short, -*] (9.5,2)
			(opamp.out) to [short, -*] (12.5,2.5);
		\draw (12.5,2.5) to [short] (13.5,2.5);
		\draw (13.5,0) to [open, v<=$v_o$, o-o] (13.5,2.5);
	\end{circuitikz}
\end{figure}
We now design the high pass stage.\\
We will use the same normalised Butterworth polynomial as from the low pass design section.\\
Therefore, the transfer function of the 3$^\text{rd}$ order normalised high pass Butterworth filter is:
\begin{equation*}
	\frac{s^3}{\, \big(s + 1 \big) \big(s^2 + s + 1 \big)} = \frac{s}{\, s+1 \,} \cdot \frac{s^2}{\, s^2 + s + 1 \,}
\end{equation*}
This results in the cascade of a 1$^\text{st}$ order high pass Butterworth filter with a 2$^\text{nd}$ order high pass Butterworth filter, both with passband gains of 0 dB and cutoff frequencies of 1 rad/s. This can be achieved with the following circuit:
\begin{figure}[H]
	\centering
	\begin{circuitikz}
		\draw (0,2.5) to [R = $R_1$] (2,2.5)
			to [C = $C_1$] (3.5,2.5)
			to [short] (3.5,4)
			to [R = $R_2$] (6.5,4)
			to [short] (6.5,2);
		\draw (0,0) to [short] (14.5,0);
		\draw (0,2.5) to [open, v=$v_i$, o-o] (0,0);
		\draw (5,2) node[op amp] (opamp) {}
			(opamp.-) to [short, -*] (3.5,2.5)
			(opamp.+) to [short] (3.5,1.5)
				to [short, -*] (3.5,0)
			(opamp.out) to [short, -*] (6.5,2);
		\draw (6.5,2) to [C = $C_2$] (8.5,2)
			to [C = $C_3$] (10.5,2)
			to [R = $R_4$, -*] (10.5,0);
		\draw (8.5,2) to [short, *-] (8.5,2.5)
			to [R=$R_3$] (8.5,4.5)
			to [short] (10.5,4.5)
			to [short] (10.5,3);
		\draw (10.5,4.5) to [short, *-] (13.5,4.5)
			to [short] (13.5,2.5);
		\draw (12,2.5) node[op amp] (opamp) {}
			(opamp.-) to [short] (10.5,3)
			(opamp.+) to [short, -*] (10.5,2)
			(opamp.out) to [short, -*] (13.5,2.5);
		\draw (13.5,2.5) to [short] (14.5,2.5);
		\draw (14.5,0) to [open, v<=$v_o$, o-o] (14.5,2.5);
	\end{circuitikz}
\end{figure}
We need to find resistor and capacitor values for the normalised filters.\\
\\
Let $C_1 = C_2 = C_3 = 1$ F.\\
For the normalised 1$^\text{st}$ order high pass Butterworth filter, we note that the transfer function is the same as the transfer function of the prototypical 1$^\text{st}$ order active high pass filter. Therefore, the component values for this normalised Butterworth filter should match those of the 1$^\text{st}$ order prototype filter, and so $R_1 = R_2 = 1 \, \Omega$.\\
\\
For the normalised 2$^\text{nd}$ order high pass Butterworth filter, we note that the transfer function is:
\begin{align*}
	H_n(s) &= \frac{s^2/ (R_3 \, R_4)}{s^2 + \frac{2}{\, R_3 \,} \cdot s + \frac{1}{\, R_3 \, R_4 \,}} \\
	&= \frac{s^2}{\, s^2 + s + 1 \,} \\
	\\
	\therefore \frac{2}{R_3} &= 1 \\
	\therefore R_3 &= 2 \, \Omega \\
	\\
	\therefore \frac{1}{\, R_3 \, R_4 \,} &= 1 \\
	\therefore R_4 &= \frac{1}{R_3} \\
	&= 0.5 \, \Omega
\end{align*}
With these above values, we have a 3$^\text{rd}$ order high pass Butterworth filter with a gain of 0 dB in the passband and a cutoff frequency of 1 rad/s.\\
\\
We now need to adjust the filter to get the desired cutoff frequency of 5 kHz.\\
First off we perform frequency scaling by finding the frequency scaling factor $k_f$:
\begin{align*}
	k_f &= \frac{\omega_c^{'}}{\omega_c} \\
	&= \frac{2 \cdot \pi \cdot 5000}{1} \\
	&= 31,415.9
\end{align*}
Now we must use 10 nF capacitors in the final design of the high pass filter section, so we perform magnitude scaling by finding the magnitude scaling factor $k_m$:
\begin{align*}
	k_m &\triangleq \frac{C}{\, C^{'} \, k_f \,} \\
	&= \frac{1}{10^{-8} \cdot 31415.9} \\
	&= 3183.1
\end{align*}
We can now scale the resistors to new values to achieve the desired cutoff frequency:
\begin{align*}
	R_n^{'} &= R_n \cdot k_m \\
	\\
	\therefore R_1^{'} &= 1 \cdot 3183.1 = 3.18 \, \text{k}\Omega \\
	\therefore R_2^{'} &= 1 \cdot 3183.1 = 3.18 \, \text{k}\Omega \\
	\therefore R_3^{'} &= 2 \cdot 3183.1 = 6.37 \, \text{k}\Omega \\
	\therefore R_4^{'} &= 0.5 \cdot 3183.1 = 1.59 \, \text{k}\Omega
\end{align*}
And so we have finished the design of the high pass section, whose circuit is shown below:
\begin{figure}[H]
	\centering
	\begin{circuitikz}
		\draw (0,2.5) to [R = $3.18 \, \text{k}\Omega$] (2,2.5)
			to [C = $10 \, \text{nF}$] (3.5,2.5)
			to [short] (3.5,4)
			to [R = $3.18 \, \text{k}\Omega$] (6.5,4)
			to [short] (6.5,2);
		\draw (0,0) to [short] (14.5,0);
		\draw (0,2.5) to [open, v=$v_i$, o-o] (0,0);
		\draw (5,2) node[op amp] (opamp) {}
			(opamp.-) to [short, -*] (3.5,2.5)
			(opamp.+) to [short] (3.5,1.5)
				to [short, -*] (3.5,0)
			(opamp.out) to [short, -*] (6.5,2);
		\draw (6.5,2) to [C = $10 \, \text{nF}$] (8.5,2)
			to [C = $10 \, \text{nF}$] (10.5,2)
			to [R = $1.59 \, \text{k}\Omega$, -*] (10.5,0);
		\draw (8.5,2) to [short, *-] (8.5,2.5)
			to [R=$6.37 \, \text{k}\Omega$] (8.5,4.5)
			to [short] (10.5,4.5)
			to [short] (10.5,3);
		\draw (10.5,4.5) to [short, *-] (13.5,4.5)
			to [short] (13.5,2.5);
		\draw (12,2.5) node[op amp] (opamp) {}
			(opamp.-) to [short] (10.5,3)
			(opamp.+) to [short, -*] (10.5,2)
			(opamp.out) to [short, -*] (13.5,2.5);
		\draw (13.5,2.5) to [short] (14.5,2.5);
		\draw (14.5,0) to [open, v<=$v_o$, o-o] (14.5,2.5);
	\end{circuitikz}
\end{figure}
We now design the gain stage.\\
We are designing based off a desired magnitude response, with no attention paid to the phase response. Therefore, we will choose to implement the gain using an inverting summing op-amp amplifier circuit.
\begin{figure}[H]
	\centering
	\begin{circuitikz}
		\draw (0,3.5) to [R=$R_s$] (3,3.5)
			to [short] (3,2.5);
		\draw (0,3.5) to [open, v=$v_a$, o-] (0,0);
		\draw (1,1.5) to [R=$R_s$] (3,1.5)
			to [short] (3,2.5);
		\draw (1,1.5) to [open, v=$v_b$, o-*] (1,0);
		\draw (0,0) to [short, o-o] (7.5,0);
		\draw (3,2.5) to [short, *-] (3.5,2.5)
			to [short] (3.5,4)
			to [R=$R_f$] (6.5,4)
			to [short] (6.5,2);
		\draw (5,2) node[op amp] (opamp) {}
			(opamp.-) to [short, -*] (3.5,2.5)
			(opamp.+) to [short] (3.5,1.5)
				to [short, -*] (3.5,0)
			(opamp.out) to [short, -*] (6.5,2)
				to [short, -o] (7.5,2);
		\draw (7.5,0) to [open, v<=$v_o$] (7.5,2);
	\end{circuitikz}
\end{figure} 
We note that there is equal weighting to both inputs, and so the transfer function for either input is:
\begin{equation*}
	H(s) = -\frac{R_f}{R_s}
\end{equation*}
Now we want a passband gain of 20 dB $=$ 10 V/V. And so we want:
\begin{align*}
	\big|H(s)\big| &= 10 \\
	\therefore \frac{R_f}{R_s} &= 10, \qquad \text{let $R_s$ = 1 k$\Omega$,} \implies R_f = 10 \ \text{k}\Omega
\end{align*}
We now have all of the separate stages designed, so we bring them together as described already:
\begin{figure}[H]
	\centering
	\scalebox{0.65}{
		\begin{circuitikz}
			%input
			\draw (0,0) to [short] (-2,0)
				to [vsource, v_=$v_i$] (-2,-2);
			\draw (-2,-2) node[ground] {};

			%low pass
			\draw (0,0) to [short] (0,3.5)
				to [R=$1 \, \text{k}\Omega$] (2,3.5)
				to [short, *-] (2,5)
				to [R=$1 \, \text{k}\Omega$] (5,5)
				to [short] (5,3);
			\draw (2,5) to [short, *-] (2,6.5)
				to [C=$318 \, \text{nF}$] (5,6.5)
				to [short, -*] (5,5);
			\draw (3.5,3) node[op amp] (opamp) {}
				(opamp.-) to [short, -*] (2,3.5)
				(opamp.+) to [short] (2,2.5)
					to [short] (2,1)
				(opamp.out) to [short, -*] (5,3);
			\draw (2,1) node[ground] {};
			\draw (5,3) to [R = $1 \, \text{k}\Omega$] (7.5,3)
				to [R=$1 \, \text{k}\Omega$] (9.5,3)
				to [C=$159 \, \text{nF}$] (9.5,1);
			\draw (9.5,1) node[ground] {};
			\draw (7.5,3) to [C=$637 \, \text{nF}$, *-] (7.5,5.5)
				to [short] (9.5,5.5)
				to [short] (9.5,4);
			\draw (9.5,5.5) to [short, *-] (12.5,5.5)
				to [short] (12.5,3.5);
			\draw (11,3.5) node[op amp] (opamp) {}
				(opamp.-) to [short] (9.5,4)
				(opamp.+) to [short, -*] (9.5,3)
				(opamp.out) to [short, -*] (12.5,3.5);
			\draw (12.5,3.5) to [short] (13.5,3.5)
				to [R=$1 \, \text{k}\Omega$] (16,3.5)
				to [short] (16,0);

			%high pass
			\draw (0,0) to [short, *-] (0,-3.5)
				to [R = $3.18 \, \text{k}\Omega$] (2,-3.5)
				to [C = $10 \, \text{nF}$] (3.5,-3.5)
				to [short] (3.5,-2)
				to [R = $3.18 \, \text{k}\Omega$] (6.5,-2)
				to [short] (6.5,-4);
			\draw (5,-4) node[op amp] (opamp) {}
				(opamp.-) to [short, -*] (3.5,-3.5)
				(opamp.+) to [short] (3.5,-4.5)
					to [short] (3.5,-6)
				(opamp.out) to [short, -*] (6.5,-4);
			\draw (3.5,-6) node[ground] {};
			\draw (6.5,-4) to [C = $10 \, \text{nF}$] (8.5,-4)
				to [C = $10 \, \text{nF}$] (10.5,-4)
				to [R = $1.59 \, \text{k}\Omega$] (10.5,-6);
			\draw (10.5,-6) node[ground] {};
			\draw (8.5,-4) to [short, *-] (8.5,-3.5)
				to [R=$6.37 \, \text{k}\Omega$] (8.5,-1.5)
				to [short] (10.5,-1.5)
				to [short] (10.5,-3);
			\draw (10.5,-1.5) to [short, *-] (13.5,-1.5)
				to [short] (13.5,-3.5);
			\draw (12,-3.5) node[op amp] (opamp) {}
				(opamp.-) to [short] (10.5,-3)
				(opamp.+) to [short, -*] (10.5,-4)
				(opamp.out) to [short, -*] (13.5,-3.5);
			\draw (13.5,-3.5) to [R=$1 \, \text{k}\Omega$] (16,-3.5)
				to [short] (16,0);

			%gain
			\draw (16,0) to [short, *-] (17,0)
				to [short] (17,1.5)
				to [R=$10 \, \text{k}\Omega$] (20,1.5)
				to [short] (20,-0.5);
			\draw (18.5,-0.5) node[op amp] (opamp) {}
				(opamp.-) to [short, -*] (17,0)
				(opamp.+) to [short] (17,-1)
					to [short] (17,-2)
				(opamp.out) to [short, -*] (20,-0.5)
					to [short, -*] (21,-0.5);
			\draw (17,-2) node[ground] {};
			\draw (21,-2) node[ground] {};
			\draw (21,-2) to [short,-*] (21,-2);
			\draw (21,-2) to [open, v<=$v_o$] (21,-0.5);
		\end{circuitikz}
	}
\end{figure}
And so we have a 3$^\text{rd}$ order bandreject Butterworth filter with passgand gain of 20 dB, a lower cutoff frequency of 500 Hz and an upper cutoff frequency of 5 kHz, using only 1 k$\Omega$ resistors in the low pass stage and 10 nF capacitors in the high pass stage.
\\
