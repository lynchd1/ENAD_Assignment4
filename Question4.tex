%Question 4

Design this bandreject filter by connecting a low pass Butterworth filter with a cutoff frequency of 500 Hz and gain of 0 dB in parallel with a high pass Butterworth filter with a cutoff frequency of 5 kHz and gain of 0 dB (thereby rejecting frequencies between 500 Hz and 5 kHz), then connecting both to separate inputs of a summing amplifier with gain of 20 dB.\\
\\
Start with designing the low pass section.\\
The 3$^\text{rd}$ order normalised Butterworth polynomial is:
\begin{equation*}
	\big(s+1 \big) \big(s^2+s+1 \big)
\end{equation*}
Therefore the 3$^\text{rd}$ order normalised low pass Butterworth filter has the following transfer function:
\begin{equation*}
	\frac{1}{\big(s+1 \big) \big(s^2+s+1 \big)} = \frac{1}{\, s+1 \,} \cdot \frac{1}{\, s^2+s+1 \,}
\end{equation*}
This results in a cascade of a 1$^\text{st}$ order low pass Butterworth filter with a 2$^\text{nd}$ order low pass Butterworth filter, both with passband gains of 0 dB and cutoff frequencies of 1 rad/s. This can be achieved with the following circuit:
\begin{figure}[H]
	\centering
	\begin{circuitikz}
		\draw (0,2.5) to [R=$R_1$] (2,2.5)
			to [short] (2,4)
			to [R=$R_2$] (5,4)
			to [short] (5,2);
		\draw (2,4) to [short, *-] (2,5.5)
			to [C=$C_1$] (5,5.5)
			to [short, -*] (5,4);
		\draw (0,0) to [short] (13.5,0);
		\draw (0,2.5) to [open, v=$v_i$, o-o] (0,0);
		\draw (3.5,2) node[op amp] (opamp) {}
			(opamp.-) to [short, -*] (2,2.5)
			(opamp.+) to [short] (2,1.5)
				to [short, -*] (2,0)
			(opamp.out) to [short, -*] (5,2);
		\draw (5,2) to [short] (5.5,2)
			to [R = $R_3$] (7.5,2)
			to [R=$R_4$] (9.5,2)
			to [C=$C_3$, -*] (9.5,0);
		\draw (7.5,2) to [C=$C_2$, *-] (7.5,4.5)
			to [short] (9.5,4.5)
			to [short] (9.5,3);
		\draw (9.5,4.5) to [short, *-] (12.5,4.5)
			to [short] (12.5,2.5);
		\draw (11,2.5) node[op amp] (opamp) {}
			(opamp.-) to [short] (9.5,3)
			(opamp.+) to [short, -*] (9.5,2)
			(opamp.out) to [short, -*] (12.5,2.5);
		\draw (12.5,2.5) to [short] (13.5,2.5);
		\draw (13.5,0) to [open, v<=$v_o$, o-o] (13.5,2.5);
	\end{circuitikz}
\end{figure}
We need to find resistor and capacitor values for the normalised filters.\\
\\
Let $R_1 = R_2 = R_3 = R_4 = 1 \Omega$.\\
For the normalised 1$^\text{st}$ order low pass Butterworth filter, we note that the transfer function is the same as the transfer function of the prototypical 1$^\text{st}$ order active low pass filter. Therefore $C_1 = 1$ F.\\
\\
For the normalised 2$^\text{nd}$ order low pass Butterworth filter, we note that the transfer function is:
\begin{align*}
	H_n(s) &= \frac{1/(C_2 \, C_3)}{s^2 + \frac{2}{C_2} \cdot s + \frac{1}{\, C_2 \, C_3 \,}} \\
	&= \frac{1}{\, s^2 + s + 1 \,} \\
	\\
	\therefore \frac{2}{C_2} &= 1 \\
	\therefore C_2 &= 2 \, \text{F} \\
	\\
	\therefore \frac{1}{C_2 \, C_3} &= 1 \\
	\therefore C_3 &= \frac{1}{C_2} \\
	&= 0.5 \, \text{F}
\end{align*}
With these above values, we have a 3$^\text{rd}$ order low pass Butterworth filter with a gain of 0 dB in the passband and a cutoff frequency of 1 rad/s.\\
\\
We now need to adjust the filter to get the desired cutoff frequency of 500 Hz (desired gain will be achieved by the gain stage).\\
First off we perform frequency scaling by finding the frequency scaling factor $k_f$:
\begin{align*}
	k_f &\triangleq \frac{\omega_c^{'}}{\omega_c} \\
	&= \frac{2 \cdot \pi \cdot 500}{1} \\
	&= 3141.6
\end{align*}
Now we must use 1 k$\Omega$ resistors in the low pass filter, so we perform magnitude scaling by finding the magnitude scaling factor $k_m$:
\begin{align*}
	k_m &\triangleq \frac{R^{'}}{R} \\
	&= \frac{1000}{1} \\
	&= 1000
\end{align*}
We can now scale the capacitors to new values to achieve the desired cutoff frequency:
\begin{align*}
	C_n^{'} = \frac{C_n}{\, k_m \, k_f \,} \\
	\\
	\therefore C_1^{'} &= \frac{1}{1000 \cdot 3141.6} = 318 \, \text{nF} \\
	\therefore C_2^{'} &= \frac{2}{1000 \cdot 3141.6} = 637 \, \text{nF} \\
	\therefore C_3^{'} &= \frac{0.5}{1000 \cdot 3141.6} = 159 \, \text{nF}
\end{align*}
And so we have finished the design of the 3$^\text{rd}$ order low pass Butterworth filter, whose circuit is shown below:
\begin{figure}[H]
	\centering
	\begin{circuitikz}
		\draw (0,2.5) to [R=$1 \, \text{k}\Omega$] (2,2.5)
			to [short] (2,4)
			to [R=$1 \, \text{k}\Omega$] (5,4)
			to [short] (5,2);
		\draw (2,4) to [short, *-] (2,5.5)
			to [C=$318 \, \text{nF}$] (5,5.5)
			to [short, -*] (5,4);
		\draw (0,0) to [short] (13.5,0);
		\draw (0,2.5) to [open, v=$v_i$, o-o] (0,0);
		\draw (3.5,2) node[op amp] (opamp) {}
			(opamp.-) to [short, -*] (2,2.5)
			(opamp.+) to [short] (2,1.5)
				to [short, -*] (2,0)
			(opamp.out) to [short, -*] (5,2);
		\draw (5,2) to [short] (5.5,2)
			to [R = $1 \, \text{k}\Omega$] (7.5,2)
			to [R=$1 \, \text{k}\Omega$] (9.5,2)
			to [C=$159 \, \text{nF}$, -*] (9.5,0);
		\draw (7.5,2) to [C=$637 \, \text{nF}$, *-] (7.5,4.5)
			to [short] (9.5,4.5)
			to [short] (9.5,3);
		\draw (9.5,4.5) to [short, *-] (12.5,4.5)
			to [short] (12.5,2.5);
		\draw (11,2.5) node[op amp] (opamp) {}
			(opamp.-) to [short] (9.5,3)
			(opamp.+) to [short, -*] (9.5,2)
			(opamp.out) to [short, -*] (12.5,2.5);
		\draw (12.5,2.5) to [short] (13.5,2.5);
		\draw (13.5,0) to [open, v<=$v_o$, o-o] (13.5,2.5);
	\end{circuitikz}
\end{figure}
We now design the high pass section.\\

